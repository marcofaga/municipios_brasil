%Modelo Simples para documentação e reports simples.
%Feito por Marco Antonio Faganello
%Data: 04 de julho de 2018
%Usando Xelatex para compilar

\documentclass[12pt, portuguese]{article}
\usepackage[a4paper,bindingoffset=0.2in,%
            left=1in,right=1in,top=1in,bottom=1in,%
            footskip=.25in]{geometry}
\usepackage{graphicx}
\usepackage{threeparttable}
\usepackage{tabularx}
\usepackage{multirow}
\usepackage{ltxtable}

%encoding
%--------------------------------------
\usepackage[utf8]{inputenc}
\usepackage[T1]{fontenc}
%--------------------------------------

%Portuguese-specific commands
%--------------------------------------
\usepackage[portuguese]{babel}
%--------------------------------------

%Hyphenation rules
%--------------------------------------
\usepackage{hyphenat}
\hyphenation{mate-mática recu-perar}

\begin{document}

\title{%
  HistMun - Municípios do Brasil (1995 - 2017) \\
  \large  Banco de dados com estatísticas básicas sobre os municípios brasileiros em uma série histórica \\
  v1.0}

\author{Marco Antonio Faganello}

\maketitle

\section{Introdução}

Esse projeto compreende a construção de um banco de dados contendo estatísticas básicas sobre os municípios brasileiros dentro de uma série histórica. A princípio desde os anos de 1995 até 2017. O projeto está em sua primeira versão e contém apenas os dados de contagem da população nos anos citados. A ideia é que o banco possa auxiliar os pesquisadores oferecendo acesso a dados básicos, usados em grande parte dos projetos, de maneira rápida e condensada.

\section{Arquivos}

O banco está disponível em formato RData e csv e podem ser acessados pelo link: https://github.com/marcofaga/municipios\_brasil \\

\noindent
Arquivos principais:\\
histmun.RData - Banco de dados em formato RData (R)\\
histmun.csv - Banco de dados em formato csv\\
script\_histmun.R - Script R de criação do banco histmun.RData\\
dicionario.csv - dicionário do banco de dados em formato csv\\

\section{Dicionário}

\LTXtable{\textwidth}{tabdic.tex}

\end{document}
